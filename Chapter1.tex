\documentclass[12pt]{report}
\usepackage[utf8]{inputenc}
\usepackage[a4paper, total={6.5in,10in}]{geometry}
\setlength{\parskip}{.75em}
\setlength{\parindent}{0em}
\usepackage{amsmath}
\title{Solutions Attempt for Eisberg and Resnick's QM}
\author{Angel-Emilio Villegas Sanchez}
\date{May 2021}

\begin{document}

\maketitle
\chapter*{Introduction}
The following set of answers is by a recently graduated student of Physics. These should be considered as possible  solutions, but by no means are meant to be complete - or even correct! This is mainly an exercise by me, Angel Villegas, at sharing Physics knowledge through the internet, with the hope of collaboration with other students who also seek to establish and discuss answers to Physics problems.   
\chapter{Thermal Radiation and Planck's Postulate}

\section*{Questions:}
\subsection*{1) Does a blackbody always appear black? Explain the term blackbody}
Eisberg and Resnick define \textit{blackbodies} as "bodies that have surfaces which absorb all the thermal radiation incident upon them . . . such bodies do not reflect light and appear black when their temperatures are low enough that they are not self-luminous" (Pg. 3). The answer to the question would be \textbf{no}; blackbodies do not always appear black since they could be self-luminous. Stars, like the sun, are not perfect blackbodies, but the spectrum they radiate approximate quite nicely to that of a blackbody since most of the energy radiated comes from within, and it is not of reflected radiation. If the object wasn't self-luminous, then it would most certainly look as black as The Void!
\subsection*{2) Pockets formed by coals in a coal fire seem brighter than the coals themselves. Is the temperature in such pocket appreciably higher than the surface temperature of an exposed glowing coal?}
The seemingly bright pockets should be hotter than the exposed surface of a coal since the pockets act as a type of blackbody cavity (described on Pg. 5) that absorbs and reflects the radiation emitted by the coals. The answer to this question might be more complex depending on what we consider the "pocket"
\begin{itemize}
    \item If the "pocket" is the air inside, then this air must come into equilibrium with the walls of the cavity formed by the coals. However, assuming there is air, then that means this same air draws temperature away from the exposed coal surface.
    \item If there's no air, then I strongly believe there might be no glow, and there would be no air to reduce the temperature of the exposed coal surface. Thus, in a vacuum both would be at the same temperature (though I am not sure if they could combust at all in a vacuum).
\end{itemize}
\subsection*{3) If we look into a cavity whose walls are kept at a constant temperature no details of the interior are visible. Explain.}
This is an oddly-phrased question. If seeing darkness is what they mean by, "no details of the interior are visible," then, assuming there's no self-luminosity, this cavity acts as a \textit{blackbody} since all the inner walls should absorb the radiation within. Like other blackbodies, reflected light is not possible.

\section*{Problems:}
\newcommand{\lambmax}{\lambda_{max}}
\newcommand{\Wiens}{2.898\,\cdot\,10^{-3}\;}
\subsection*{1) At what wavelength does a cavity at \(6000^\circ K\) radiate most per unit wavelength?}
We can use Wien's displacement law, \(\nu_{max} \propto \;T\), written in the form, 
\[\lambmax T = const\]
where Wien's constant is \(\Wiens m^\circ K\). All we have to is find the variable \(\lambmax\),
\[\lambmax = \frac{\Wiens}{6000} = 483 \cdot 10^{-9}\; m = 483\; nm \]
\newcommand{\specrad}{R_T(\nu)}
\newcommand{\enerden}{\rho_T(\nu)}
\newcommand{\phE}{h\nu}
\newcommand{\molE}{kT}
\subsection*{2) Show that the proportionality constant in \(\enerden \propto \specrad\) is \(4/c\). That is, show that the relation between spectral radiancy \(\specrad\) and energy density \(\enerden\) is \(\specrad\,d\nu = (c/4)\enerden d\nu\)}
In order to find this constant of proportionality I will arrange the equation in the following manner 
\[\int_0^{\infty}{\specrad\;d\nu} = Z\int_0^{\infty}{\enerden\;d\nu}\]

The left integral has been found empirically to be \(\sigma T^4\) (\textbf{Stefan's Law}). The differential for the energy density in a unit volume is given by \textit{Planck's blackbody spectrum} formula
\[\enerden d\nu = \frac{8 \pi \nu^2}{c^3}\;\frac{\phE}{e^{\phE /\molE} - 1}\; d\nu\]

This formula is derived in Pg.6 - Pg.17. It requires a good grasp of integration, a decent grasp of E\&M, and some concepts of Statistical Mechanics (particularly the Boltzmann Distribution). After some research, the integral on the right requires sophisticated integration skills in order to find the correct answer analytically. The following set-up has been copied from a Slader solution by Abdelrahman Almeghari on slader,
\[\int_0^\infty \frac{8 \pi \nu^2}{c^3}\;\frac{\phE}{e^{\phE /\molE} - 1}\; d\nu\]
Substitute \(\phE/\molE\) with \(x\), then
\[\int_0^\infty \frac{8 \pi \nu^2}{c^3}\;\frac{kTx}{e^{x} - 1}\; \frac{\molE}{h}\; dx\]
and also substitute \(\nu^2\)
\[\frac{8 \pi}{c^3}\int_0^\infty\;(\frac{kTx}{h})^2\;\frac{kTx}{e^{x} - 1}\; \frac{\molE}{h}\; dx\]
\[\frac{8 \pi}{c^3}\;\frac{(kT)^4}{h^3}\int_0^\infty\frac{x^3}{e^{x} - 1}\;dx\]

We are left with a very difficult problem that, unless you've coursed through some nice pure mathematics courses, or a really good Physics program, then you won't be able to figure it out. There seems to be more than one potential solution: one involves using \textbf{contour integration}, the other quick and dirty method involves using a \textbf{Riemann zeta function}. Here's the integral form of the Riemann Zeta function
\[\zeta(s) = \frac{1}{\Gamma(s)}\int_0^\infty\frac{x^{s-1}}{e^x - 1}\]
You will notice it matches quite perfectly the 
\end{document} 
